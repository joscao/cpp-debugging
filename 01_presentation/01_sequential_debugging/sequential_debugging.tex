\documentclass[aspectratio=1610]{beamer}
%[aspectratio=169]
% ------------------------------------------------------------------------
% PRAESENTATIONSAUSWAHL
\usecolortheme[RGB={3,138,94}]{structure} 
\mode<presentation> {
	%\usetheme{CambridgeUS} 
	\usetheme{Warsaw} 
	% oder Antibes, Bergen, Berkeley, ....
	
	%\setbeamercovered{transparent}
	
}
\useoutertheme{infolines}
\setbeamertemplate{headline}{}

\beamertemplatenavigationsymbolsempty
\definecolor{unigruen}{RGB}{3,138,94} 

\usefonttheme{professionalfonts}  % Aenderung der Schriftart
% (bei mathematischen Ausdruecken)
% ------------------------------------------------------------------------


% ------------------------------------------------------------------------
% EINGEBUNDENE PAKETE
\usepackage{xcolor}
\usepackage{varwidth}
\usepackage[utf8]{inputenc}
\usepackage{times,relsize,xspace}
\usepackage[T1]{fontenc}
\usepackage{mathtools}
\usepackage{dirtytalk}
\usepackage{minted,hyperref}


\usepackage[style=authortitle,]{biblatex} %Imports biblatex package
\addbibresource{bibliography.bib} %Import the bibliography file
% Oder was auch immer. Zu beachten ist, das Font und Encoding passen
% muessen. Falls T1 nicht funktioniert, kann man versuchen, die Zeile
% mit fontenc zu loeschen.
% ------------------------------------------------------------------------

\newtheorem{thm}{Theorem}[theorem]

% ------------------------------------------------------------------------

\newcommand{\Rplus}{\protect\hspace{-.1em}\protect\raisebox{.35ex}{\smaller{\smaller\textbf{+}}}}
\newcommand{\Cpp}{\mbox{C\Rplus\Rplus}\xspace}

% Define the centeredblock environment to accept an optional argument for width
\newenvironment{centeredblock}[2][0.8\textwidth]
{ % This code will be executed at the beginning of the environment
	\begin{center}
		\begin{varwidth}{#1} % Use the argument for width
			\begin{block}{#2}
				\centering
			}
			{ % This code will be executed at the end of the environment
			\end{block}
		\end{varwidth}
	\end{center}
}


\title{Sequential and Parallel Debugging}
\subtitle{}
\author{Jonathan Schmalfuß}
\institute[UBT]{Chair of Scientific Computing \\ University of Bayreuth}
\date{\today}
%\tiny A talk adapt from \href{hackingcpp.com}{hackingcpp}

\begin{document}
	
	% TITELSEITE
	{
		\setbeamertemplate{headline}{} % Bei der Titelseite keine Kopf- und
		\setbeamertemplate{footline}{} % Fusszeile
		\begin{frame}
			\pagestyle{empty}
			\titlepage
			\pagestyle{empty}
		\end{frame}
	}
	
	\logo{} % remove logo from all following pages
	% Titelseite bei der Nummerierung ignorieren
	\addtocounter{framenumber}{-1}
	\setminted{fontsize=\footnotesize}
	
	\begin{frame}[fragile]{Sequential debugging}
		\begin{centeredblock}{}
			\say{Debugging is twice as hard as writing the code in the first place. Therefore, if you write the code as cleverly as possible, you are, by definition, not smart enough to debug it.} - Brian Kernighan, professor at Princeton University.
		\end{centeredblock}
	\end{frame}
	
	\begin{frame}[fragile]{Compiler Flags}
		\begin{centeredblock}{}
			\href{https://github.com/cpp-best-practices/cppbestpractices/blob/master/02-Use_the_Tools_Available.md#gcc--clang}{Recommended Flags by Jason Turner}
		\end{centeredblock}
	\end{frame}
	
	\begin{frame}[fragile]{GNU command line debugger: gdb}
		\begin{columns}
			\begin{column}{0.375\textwidth}
				\begin{block}{}
					\inputminted[firstline=0,lastline=14]{cpp}{code/gdb_example_sum_hackingcpp.cpp}
				\end{block}
			\end{column}
			\hfill
			\begin{column}{0.62\textwidth}
				\begin{block}{}
					\inputminted[firstline=15,lastline=24]{cpp}{code/gdb_example_sum_hackingcpp.cpp}
				\end{block}
			\end{column}
		\end{columns}

		\begin{centeredblock}[0.75 \textwidth]{}
			Live session inspired by \href{https://hackingcpp.com/cpp/tools/gdb_intro.html}{hackingcpp: Debugging With gdb}
		\end{centeredblock}
	\end{frame}
	
	\begin{frame}[fragile]{GNU command line debugger: gdb}
		\begin{block}{}
			\begin{table}[h!]
				\small
				\centering
				\begin{tabular}{|l|l|p{7.5cm}|}
					\hline
					\textbf{key / command}      & \textbf{short form} & \textbf{meaning}                                  \\ \hline
					\texttt{<Enter>}            &                     & repeat previous command                           \\ \hline
					\texttt{<Tab>}              &                     & complete command or function name                \\ \hline
					\texttt{run [<arg>...]}     & \texttt{r [<arg>...]} & run program (with command line argument(s))       \\ \hline
					\texttt{break <loc>}        & \texttt{b <loc>}    & set breakpoint at beginning of function or at a specific line \\ \hline
					\texttt{step}               & \texttt{s}          & next instruction, step into function             \\ \hline
					\texttt{next}               & \texttt{n}          & next instruction, step over function             \\ \hline
					\texttt{jump <loc>}         & \texttt{j <loc>}    & jump to location (useful for exiting long/endless loops) \\ \hline
					\texttt{continue}           & \texttt{c}          & continue until next breakpoint or end of program \\ \hline
					\texttt{until <loc>}        & \texttt{u <loc>}    & continue until location (function /line)         \\ \hline
					\texttt{finish}             & \texttt{fin}        & finish (step out of) current function            \\ \hline
					\texttt{print <expression>} & \texttt{p}          & print value of expression, e.g., variable        \\ \hline
					\texttt{info breakpoints}   & \texttt{i b}        & list all breakpoints                             \\ \hline
					\texttt{info locals}        & \texttt{i locals}   & list local variables and their values            \\ \hline
					\texttt{backtrace}          & \texttt{bt}         & show call stack                                  \\ \hline
				\end{tabular}
				\caption{Useful gdb Commands}
				\label{table:gdb_commands}
				\vspace{-0.5cm}
			\end{table}
		\end{block}
	\end{frame}
	
	\begin{frame}[fragile]{Helpful compilation flags: g++ compiler}
		\begin{centeredblock}[0.85 \textwidth]{Warning flags}
			\begin{itemize}
				\item \texttt{-Wall}: Enables many warning flags
				\item \texttt{-Werror}: Converts any warning into an error
				\item \texttt{-Wextra}/\texttt{-W}: Additional warnings not in \texttt{-Wall}
				\item \texttt{-pedantic}/\texttt{-Wpedantic}: ISO standard compliance warnings
				\item Specific warnings:
				\begin{itemize}
					\item \texttt{-Wconversion}, \texttt{-Wcast-align}, \texttt{-Wunused}, \texttt{-Wshadow}, \texttt{-Wold-style-cast}
					\item \texttt{-Wpointer-arith}, \texttt{-Wcast-qual}, \texttt{-Wmissing-prototypes}, \texttt{-Wno-missing-braces}
				\end{itemize}
			\end{itemize}
		\end{centeredblock}
	
		{\def\thefootnote{}\footnotetext{\href{https://caiorss.github.io/C-Cpp-Notes/compiler-flags-options.html}{CPP/C++ Compiler Flags and Options}}}
	\end{frame}
	
	\begin{frame}[fragile]{Valgrind's tool suite}
		\begin{centeredblock}{}
			\begin{itemize}
				\item Memcheck: detects memory-management problems: \newline 
					\texttt{valgrind --leak-check=full --show-leak-kinds=all --track-origins=yes --verbose <myexecutable>}
				\item Helgrind: thread debugger \newline
					\texttt{valgrind --tool=helgrind <myexecutable>}
				\item Callgrind: cache profiler plus extra information about callgraphs
			\end{itemize}
		\end{centeredblock}
	\end{frame}
	
	\begin{frame}[fragile]{Sanitizers}
		\begin{centeredblock}{Sanitizers}
			\begin{itemize}
				\item AddressSanitizer: \texttt{-fsanitize=address}
				\item UndefinedBehaviorSanitizer: \texttt{-fsanitize=undefined}
				\item ThreadSanitizer: \texttt{-fsanitize=thread}
			\end{itemize}
		\end{centeredblock}
	
		{\def\thefootnote{}\footnotetext{\href{https://gcc.gnu.org/onlinedocs/gcc/Instrumentation-Options.html}{GCC compiler flags, search for fsanitize}}}
	
		\begin{centeredblock}{Honorable Mentions: Static analyzers / Linters}
			\begin{itemize}
				\item cppcheck
				\item Clang-Tidy
			\end{itemize}
		\end{centeredblock}
	\end{frame}
	
	\begin{frame}[fragile]{Time travel debugging}
		\begin{centeredblock}{}
			Idea: run gdb; stop at the point where problem occurs; walk backwards figure out what happend
			
			$\Rightarrow$ theoretically supported by gdb via commands such as \texttt{reverse-next}
			\begin{itemize}
				\item Problem: avx instruction $\Rightarrow$ unusable most of the time
				\item Solution: \texttt{rr}\footnote{\href{rr}{https://rr-project.org/}}: record a failure once, then debug the recording, deterministically, as many times as you want
			\end{itemize}
		\end{centeredblock}
	
		
	\end{frame}
	
	\begin{frame}[fragile]{Parallel debugging}
		\begin{centeredblock}{}
			\say{Sequential programming is really hard, and parallel programming is a step beyond that.} - Andrew S. Tanenbaum, professor at Vrije Universiteit Amsterdam
		\end{centeredblock}
		
		\begin{centeredblock}{}
			\say{Debugging is twice as hard as writing the code in the first place. Therefore, if you write the code as cleverly as possible, you are, by definition, not smart enough to debug it.} - Brian Kernighan, professor at Princeton University.
		\end{centeredblock}
	\end{frame}
	
	\begin{frame}[fragile]{Techniques in general}
		\begin{centeredblock}{}
			\say{Most bugs also appear in the sequential version of the code} - me 
		\end{centeredblock}
	
		\pause
		\begin{centeredblock}{Distinguish Problem}
			\begin{columns}
				\hfill
				\begin{column}{0.15 \textwidth}
					\begin{block}{}
						Sequential Problem
					\end{block}
				\end{column}
				\hfill
				$\xLeftarrow[\text{persists}]{\text{problem}}$ 
				\hfill
				\begin{column}{0.3 \textwidth}
					\begin{block}{}
						Force Sequential Execution via \texttt{MPI\_Barrier()}
					\end{block}
				\end{column}
				\hfill
				$\xRightarrow[\text{now}]{\text{works}}$
				\hfill
				\begin{column}{0.15 \textwidth}
					\begin{block}{}
						Parallel Problem
					\end{block}
				\end{column}
				\hfill
			\end{columns}
		\end{centeredblock}
		\pause
		\begin{centeredblock}[0.95 \textwidth]{The experts opinion: Anthony Williams - author/ coauthorof the thread library in \Cpp}
			\begin{enumerate}
				\item Reviewing code to locate potential bugs
				\item Locating concurrency-related bugs by testing / Designing for testability
			\end{enumerate}
		\end{centeredblock}
	\end{frame}

	\begin{frame}[fragile]{Questions to think about when reviewing multithreaded code}
		\begin{centeredblock}{}
			\begin{itemize}
				\item Which data needs to be protected from concurrent access?
				\item How do you ensure that the data is protected?
				\item Where in the code could other threads be at this time?
				\item Which mutexes does this thread hold?
				\item Which mutexes might other threads hold?
				\item Are there any ordering requirements between the operations done in this
				thread and those done in another? How are those requirements enforced?
				\item Is the data loaded by this thread still valid? Could it have been modified by
				other threads?
				\item If you assume that another thread could be modifying the data, what would that
				mean and how could you ensure that this never happens?
			\end{itemize}
		\end{centeredblock}
	\end{frame}
	
	\begin{frame}[fragile]{MPI debugging}
		\begin{centeredblock}{Debug Deadlocks: Attach to the process [Live-session]}
			Situation: you have a deadlock, i.e. your executable is stuck
			\begin{enumerate}
				\item Compile with debug flags: \texttt{mpic++ -g -Wall <file> -o <name>}
					\item Wait until stuck
					\item Figure out process id's via \texttt{top} or \texttt{ps -a | grep <name>}
					\item Attach to the process and see where you are stuck --> figure out what the problem is
			\end{enumerate}
		\end{centeredblock}
		\pause
		\begin{centeredblock}{Attaching debugger to serval instances of the executable}
			\begin{itemize}
				\item Use mpirun to launch separate instances of serial debuggers
				\item Drawback: many process, usually problematic
			\end{itemize}
			\href{https://www.open-mpi.org/faq/?category=debugging#serial-debuggers}{OpenMPI: FAQ: Debugging applications in parallel}
		\end{centeredblock}
	\end{frame}
	
	\begin{frame}[fragile]{MPI debugging: Memchecker}
		\begin{centeredblock}{}
			\begin{itemize}
				\item Requires: Open MPI 1.3 or later, and Valgrind 3.2.0 or later
				\item Otherwise: works, but with many false positives
				\item Needs to be enable at compilation state of OpenMPI, unfortunately often is not
				\item to enable locally, see \href{https://www.open-mpi.org/faq/?category=debugging#memchecker}{How can I use Memchecker} 
				\item \texttt{mpirun -np 2 valgrind <executable name>}
			\end{itemize}
		\end{centeredblock}
	\end{frame}
	
	\begin{frame}[fragile]{Exercises}
		\begin{centeredblock}[0.9 \textwidth]{}
			Brought you some programs, which you can check out. Find the errors.
			\begin{enumerate}
				\item Use pure gdb \texttt{sequential\_prefix\_sum.cpp}
				\item Use valgrind memchecker \texttt{memcheck\_example.cpp}
				\item Use address sanitizer \texttt{adress\_sanitizer\_problem.cpp}
				\item Use undefined behavior sanitizer for \texttt{rounding\_coordinates.cpp}
				\item Is it a sequential problem or parallel for \texttt{mpi\_reusing\_a\_buffer.cpp}
			\end{enumerate}
		\end{centeredblock}
	\end{frame}
\end{document}